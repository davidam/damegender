% This is samplepaper.tex, a sample chapter demonstrating the
% LLNCS macro package for Springer Computer Science proceedings;
% Version 2.20 of 2017/10/04
%
\documentclass[runningheads]{llncs}
%
\usepackage{graphicx}
\usepackage[utf8]{inputenc}
% Used for displaying a sample figure. If possible, figure files should
% be included in EPS format.
%
% If you use the hyperref package, please uncomment the following line
% to display URLs in blue roman font according to Springer's eBook style:
% \renewcommand\UrlFont{\color{blue}\rmfamily}

\begin{document}
%
\title{Writing and Comparing Gender Detection Tools\thanks{URJC - GSYC}}
%
%\titlerunning{Abbreviated paper title}
% If the paper title is too long for the running head, you can set
% an abbreviated paper title here
%
\author{David Arroyo Menéndez\inst{1}\orcidID{0000-0002-2986-5361} \and
Jesús González Barahona\inst{2}}
%
% First names are abbreviated in the running head.
% If there are more than two authors, 'et al.' is used.
%
\institute{Rey Juan Carlos University, Madrid, Spain\\
  \email{d.arrroyome@alumnos.urjc.es}\\
  \url{http://www.davidam.com} \and
  Rey Juan Carlos University, Madrid, Spain\\
  \email{jgb@gsyc.es}\\
  \url{https://gsyc.urjc.es/jgb/}}
%
\maketitle              % typeset the header of the contribution
%
\begin{abstract}
Nowadays there are various APIs to detect gender from a name. In this
paper, we offer a tool to use and compare these apis and a method to
classify male, female and unknown applying machine learning and using
a free license. The gender detection from a name is useful to make
gender studies from social networks, mailing lists, software
repositories, articles, etc.

\keywords{Gender gap \and Gender detection tools \and Software repositories.}
\end{abstract}
%
%
%
\section{Introduction}
Santamaría and Mihaljević ~\cite{10.7717/peerj-cs.156} compares
and benchmarks five name-to-gender inference services by applying them
to the classification of a test data set consisting of 7,076 manually
labeled names. We are reproducing these experiments with our software
using different apis and our own solution (DAMe Gender)

In this study we are using this dataset and performance metrics. We
are giving a way to decide about male or female in undefined
situations applying machine learning from some informative
features. So, we are researching decisions about features and machine
learning methods.

The value to detect the gender in a name using machine learning is
related with new names don't registered in census as male or female.
On situations using nicknames, new names, diminutives, ... the humans
knows the gender in an intuitive way. Lingüistic features and
statistics about male or female, countries, ... in a name could be
interesting to decide give a name to a baby.

In this moment there are a gender gap between males and females in
computer science and science in general (STEMM: Science, Technology,
Engineering, Mathematics and Medicine) ~\cite{holman2018gender}. Create
free tools and improve the current state of art allows measure and
later create policies with facts to fix the situation.

\section{State of Art}

We are reproducing Santamaría and Mihaljević
~\cite{10.7717/peerj-cs.156} comparison bringing similar
results. Generally, a good comercial solution is determined by a wide
dataset. So, the market is dominated by propietary solutions with
money to invest in good datasets. Many names is determined by the
geographic and cultural origin, it can be detected by
surnames. Classify names and surnames in strings is detemined by good
datasets, again.

\section{Underlying Technologies}

We have chosen Python free software tools with a good scientific
impact. NLTK for Natural Language Processing
~\cite{loper2002nltk}. Scikit for Machine Learning
~\cite{pedregosa2011scikit}. Numpy for Numerical Computation
~\cite{van2011numpy}. Matplotlib to visualize results
~\cite{hunter2007matplotlib}. And Perceval ~\cite{duenas2018perceval}
to retrieve information in mailing lists and repositories.

\section{Datasets and Census Open Data}

Santamaría and Mihaljević ~\cite{10.7717/peerj-cs.156} explains the
different ways to create a dataset of 7000 labeled names. In summary,
the names are retrieved from research articles and labeled by humans,
checking in websites such as Wikipedia, or scientific web pages to
decide if it's about a male, female or undefined.

Another approach is the census. The scientific value using Open Data
is give a good explanation when we are asking about the gender from a
name (number of males and females using a specific name in a country)
versus a probability created by the way explained in Santamaría and
Mihaljević ~\cite{10.7717/peerj-cs.156} or similar.

\begin{verbatim}
$ python3 main.py David --total="ine"
David gender is male
363559  males for David from INE.es
0 females for David from INE.es
\end{verbatim}

We are using census Open Data from (Spain, USA and United
Kingdom). That's a good point because the authors are acquainted with
names in both languages.

A third approach is using a dataset from a popular free software
solution. For instance, Natural Language Tool Kit (NLTK) is providing
8000 labeled english names. The classification is male or female. The
problem again is about don't retrieve data with the social science
quality of National Statistics Institutes.

We are using the census approach as base of truth because about of a
name is male or female in a geographical area. Generally, a name has a
strong weight to determine if it's a male or a female on this
way. Although, if the census is not Open Data gender guesser dataset
is a good dataset for international names.

\section{Preliminary Results}


\subsection{Machine Learning}

These results are experimental, we are improving the choosing of
features and datasets. The datasets used in this experiment are INE.es
and NLTK corpus names (this dataset is about english names). The
features used are: first letter, last letter, a, b, c, d, e, f, g, h,
i, j, k, l, m, n, o, p, q, r, s, t, u, v, w, x, y, z, vocals,
consonants, first letter, first letter vocal, last letter vocal, last
letter consonant, last letter a. We are improving the choosing of
features with Principal Component Analysis. Take a look to the
results:


\begin{table}[]
\centering
\begin{tabular}{ll}
\bf ML Algorithm & \bf Accuracy &
Support Vector Machines & 0.7049180327868853 & \\
Naive Bayes (nltk) & 0.6677501413227812 & \\
Bernoulli Naive Bayes & 0.5962408140192199 & \\
Gaussian Naive Bayes & 0.5960994912379876 & \\
Multinomial Naive Bayes & 0.5960994912379876 & \\
Stochastic Gradient Descendent & 0.5873374788015828 & \\
\end{tabular}
\caption{Machine learning algorithms accuracies}\label{tab1}
\end{table}


These results are demostrating that using Support Vector Machines
english and spanish we are reaching results similar to another
comercial solutions about gender detection tools. Our classifier is
binary (only male and female)

We expect good results using gender guesser dataset and the Open Data
census selected (Spain, USA and UK).

\section{Conclusions}

The market of gender detection tools is dominated by companies based
on payment services through APIs. This market could be changed thanks
to free software tools and open data due to give more explicative
results for the user. Although the machine learning techniques is not
new in this field, it's an incentive for researchers in computer
science create free software tools.

These advances in computer science could be giving support to study
the gender gap in repositories and mailing lists.

\section{References}

%
% ---- Bibliography ----
%
% BibTeX users should specify bibliography style 'splncs04'.
% References will then be sorted and formatted in the correct style.
%
\bibliographystyle{splncs04}
\bibliography{bibtex}

%% \begin{thebibliography}{8}

%% \bibitem{ref_article1}
%% Due{\~n}as, Santiago and Cosentino, Valerio and Robles, Gregorio and Gonzalez-Barahona, Jesus M.: Perceval: Software project data at your will. Proceedings of the 40th International Conference on Software Engineering: Companion Proceeedings \textbf 1--4 (2018)
%% \bibitem{ref_article2}
%% Holman L, Stuart-Fox D, Hauser CE: The gender gap in science: How long until women are equally represented? \textbf (2018)
%% \bibitem{ref_article3}
%% Hunter, John D.: Matplotlib: A 2D graphics environment. Computing in Science \& Engineering \textbf{9}(3) 90--95 (2007)\
%% \bibitem{ref_article4}
%% Loper, Edward and Bird, Steven: NLTK: the natural language toolkit. Association for Computational Linguistics \textbf (2002)
%% \bibitem{ref_article5}
%% Santamaría, L., Mihaljević, H.: Comparison and benchmark of name-to-gender inference services. PeerJ Computer Science \textbf{4} 1--29 (2018)
%% \bibitem{ref_article6}
%% Pedregosa, Fabian and Varoquaoux and others: Scikit-learn: Machine learning in Python. Journal of machine learning research \textbf{12} Oct. 2825--2830 (2011)
%% \bibitem{ref_article7}
%% Van Der Walt, Stefan and Colbert, S Chris and Varoquaux, Gael: The NumPy array: a structure for efficient numerical computation. Computing in Science \& Engineering \textbf{13}(2) (2011)

%% \end{thebibliography}
\end{document}
